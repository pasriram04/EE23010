% \iffalse
\let\negmedspace\undefined
\let\negthickspace\undefined
\documentclass[journal,12pt,onecolumn]{IEEEtran}
\usepackage{cite}
\usepackage{amsmath,amssymb,amsfonts,amsthm}
\usepackage{algorithmic}
\usepackage{graphicx}
\usepackage{textcomp}
\usepackage{xcolor}
\usepackage{txfonts}
\usepackage{listings}
\usepackage{enumitem}
\usepackage{mathtools}
\usepackage{gensymb}
\usepackage{comment}
\usepackage[breaklinks=true]{hyperref}
\usepackage{tkz-euclide} 
\usepackage{listings}
\usepackage{gvv}                                        
\def\inputGnumericTable{}                                 
\usepackage[latin1]{inputenc}                                
\usepackage{color}                                            
\usepackage{array}                                            
\usepackage{longtable}                                       
\usepackage{calc}                                             
\usepackage{multirow}                                         
\usepackage{hhline}                                           
\usepackage{ifthen}                                           
\usepackage{lscape}

\newtheorem{theorem}{Theorem}[section]
\newtheorem{problem}{Problem}
\newtheorem{proposition}{Proposition}[section]
\newtheorem{lemma}{Lemma}[section]
\newtheorem{corollary}[theorem]{Corollary}
\newtheorem{example}{Example}[section]
\newtheorem{definition}[problem]{Definition}
\newcommand{\BEQA}{\begin{eqnarray}}
\newcommand{\EEQA}{\end{eqnarray}}
\newcommand{\define}{\stackrel{\triangle}{=}}
\theoremstyle{remark}
\newtheorem{rem}{Remark}
\begin{document}

\bibliographystyle{IEEEtran}
\vspace{3cm}

\title{GATE: ST - 32.2023}
\author{EE22BTECH11039 - Pandrangi Aditya Sriram$^{*}$% <-this % stops a space
}
\maketitle
%\newpage
\bigskip

\renewcommand{\thefigure}{\theenumi}
\renewcommand{\thetable}{\theenumi}


\vspace{3cm}
\textbf{Question:} Let $\cbrak{X_n}_{n \geq 1}$ be a sequence of independent and identically distributed random variables each having a mean $4$ and variance $9$. If $Y_n = \frac{1}{n} \sum_{i=1}^{n} X_i$ for $n \geq 1$, then $\lim\limits_{n \to \infty} E\sbrak{\brak{\frac{Y_n - 4}{\sqrt{n}}}^2}$ (in integer) equals \rule{2cm}{0.1mm}.
\\
\solution
% \fi
For all $X_i$ which as i.i.d's, mean $\mu = 4$ and variance $\sigma^2 = 9$,
\begin{align}
    Y_n &= \frac{1}{n} \sum_{i=1}^{n} X_i\\
    \implies Y_n - 4 &= \frac{1}{n} \sum_{i=1}^{n} \brak{X_i - 4}\\
    \implies \frac{Y_n - 4}{\sqrt{n}} &= \frac{1}{n^{\frac{3}{2}}}\sum_{i=1}^{n} \brak{X_i - 4}\\
\end{align} 
As all $X_i$ are i.i.d's,
\begin{align}
    \frac{Y_n - 4}{\sqrt{n}} &= \frac{1}{n^{\frac{3}{2}}}\sum_{i=1}^{n} \brak{X_i - 4}\\
    \implies \brak{\frac{Y_n - 4}{\sqrt{n}}}^2 &= \frac{1}{n^3} \sum_{i=1}^{n} \sum_{j=1}^{n} \brak{X_i - 4}\brak{X_j - 4}\\
    \implies E\sbrak{\brak{\frac{Y_n - 4}{\sqrt{n}}}^2} &= E\sbrak{\frac{1}{n^3} \sum_{i=1}^{n} \sum_{j=1}^{n} \brak{X_i - 4}\brak{X_j - 4}}\\
    &= \frac{1}{n^3}E\sbrak{2\sum_{i=1}^{n}\sum_{j=i+1}^{n} \brak{X_i - 4}\brak{X_j - 4} + \sum_{i=1}^n \brak{X_i - 4}^2}\\
    &= \frac{1}{n^3}\brak{2\sum_{i=1}^{n} \brak{E\sbrak{X_i} - 4}\brak{E\sbrak{X_j} - 4}+ \sum_{i=1}^{n} E\sbrak{\brak{X_i - 4}^2}}\\
    &= \frac{1}{n^3}\brak{2\sum_{i=1}^{n} 0 + \sum_{i=1}^{n} \sigma^2}\\
    &= \frac{n\sigma^2}{n^3}\\
    &= \frac{9}{n^2}
\end{align}
Taking limits,
\begin{align}
    \lim\limits_{n \to \infty} E\sbrak{\brak{\frac{Y_n - 4}{\sqrt{n}}}^2}
    = \lim\limits_{n \to \infty} \frac{9}{n^2} = 0
\end{align}
\end{document}