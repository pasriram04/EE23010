% \iffalse
\let\negmedspace\undefined
\let\negthickspace\undefined
\documentclass[journal,12pt,twocolumn]{IEEEtran}
\usepackage{cite}
\usepackage{amsmath,amssymb,amsfonts,amsthm}
\usepackage{algorithmic}
\usepackage{graphicx}
\usepackage{textcomp}
\usepackage{xcolor}
\usepackage{txfonts}
\usepackage{listings}
\usepackage{enumitem}
\usepackage{mathtools}
\usepackage{gensymb}
\usepackage{comment}
\usepackage[breaklinks=true]{hyperref}
\usepackage{tkz-euclide} 
\usepackage{listings}
\usepackage{gvv}                                        
\def\inputGnumericTable{}                                 
\usepackage[latin1]{inputenc}                                
\usepackage{color}                                            
\usepackage{array}                                            
\usepackage{longtable}                                       
\usepackage{calc}                                             
\usepackage{multirow}                                         
\usepackage{hhline}                                           
\usepackage{ifthen}                                           
\usepackage{lscape}

\newtheorem{theorem}{Theorem}[section]
\newtheorem{problem}{Problem}
\newtheorem{proposition}{Proposition}[section]
\newtheorem{lemma}{Lemma}[section]
\newtheorem{corollary}[theorem]{Corollary}
\newtheorem{example}{Example}[section]
\newtheorem{definition}[problem]{Definition}
\newcommand{\BEQA}{\begin{eqnarray}}
\newcommand{\EEQA}{\end{eqnarray}}
\newcommand{\define}{\stackrel{\triangle}{=}}
\theoremstyle{remark}
\newtheorem{rem}{Remark}
\begin{document}

\bibliographystyle{IEEEtran}
\vspace{3cm}

\title{Gaussian - 9.3.19}
\author{EE22BTECH11039 - Pandrangi Aditya Sriram$^{*}$% <-this % stops a space
}
\maketitle
\newpage
\bigskip

\renewcommand{\thefigure}{\theenumi}
\renewcommand{\thetable}{\theenumi}


\vspace{3cm}
\textbf{Question:} Suppose $X$ is a binomial distribution $B\left(6,\frac{1}{2}\right)$. Show that $X=3$ is the most likely outcome.
(Hint : $P(X=3)$ is the maximum among all $P(x_i),x_i=0,1,2,3,4,5,6$)\\
\solution
%\fi
\begin{align}
    X &\sim B\brak{6,\frac{1}{2}}
\end{align}
This implies $n = 6$, $p = \frac{1}{2}$ and $p = 1 - p = \frac{1}{2}$. 
\begin{align}
    p_X(x) &= \comb{n}{x}p^xq^{n-x}\\
    &= \comb{6}{x} \brak{\frac{1}{2}}^6
\end{align}
Thus, evaluating:
\begin{align}
    p_X(0) &= \brak{\frac{1}{2}}^6\\
    p_X(1) &= 6 \brak{\frac{1}{2}}^6\\
    p_X(2) &= 15 \brak{\frac{1}{2}}^6\\
    p_X(3) &= 20 \brak{\frac{1}{2}}^6\\
    p_X(4) &= 15 \brak{\frac{1}{2}}^6\\
    p_X(5) &= 6 \brak{\frac{1}{2}}^6\\
    p_X(6) &= \brak{\frac{1}{2}}^6
\end{align}
Thus, $X = 3$ is the most likely outcome.
\end{document}