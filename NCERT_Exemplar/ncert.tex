\let\negmedspace\undefined
\let\negthickspace\undefined
\documentclass[journal,12pt,twocolumn]{IEEEtran}
\usepackage{cite}
\usepackage{amsmath,amssymb,amsfonts,amsthm}
\usepackage{algorithmic}
\usepackage{graphicx}
\usepackage{textcomp}
\usepackage{xcolor}
\usepackage{txfonts}
\usepackage{listings}
\usepackage{enumitem}
\usepackage{mathtools}
\usepackage{gensymb}
\usepackage[breaklinks=true]{hyperref}
\usepackage{tkz-euclide} 
\usepackage{listings}
\usepackage{gvv}
\newtheorem{theorem}{Theorem}[section]
\newtheorem{problem}{Problem}
\newtheorem{proposition}{Proposition}[section]
\newtheorem{lemma}{Lemma}[section]
\newtheorem{corollary}[theorem]{Corollary}
\newtheorem{example}{Example}[section]
\newtheorem{definition}[problem]{Definition}
\newcommand{\BEQA}{\begin{eqnarray}}
\newcommand{\EEQA}{\end{eqnarray}}
\newcommand{\define}{\stackrel{\triangle}{=}}
\theoremstyle{remark}
\newtheorem{rem}{Remark}
\begin{document}

\bibliographystyle{IEEEtran}
\vspace{3cm}

\title{Exemplar - 10.13.2.7}
\author{EE22BTECH11039 - Pandrangi Aditya Sriram$^{*}$% <-this % stops a space
}
\maketitle
\newpage
\bigskip

\renewcommand{\thefigure}{\theenumi}
\renewcommand{\thetable}{\theenumi}

Apoorv throws two dice once and computes the product of the numbers appearing
on the dice. Peehu throws one die and squares the number that appears on it. Who
has the better chance of getting the number 36? Why?\\\solution
Let the random variables $X$ and $Y$ represent Apoorv's dice rolls, and let event $A$ represent the product of the numbers shown on the dice being 36. Assuming all dice rolls are equally likely,
\begin{align}
\pr{X=6} &= \frac{1}{6}\\
\pr{Y=6} &= \frac{1}{6}
\end{align}
The only case leading to event $A$ is rolling 6 twice. Thus, 
\begin{align}
\pr{A} &= \pr{(X=6)\cdot(Y=6)} \label{eq:3}
\end{align}
Since both dice rolls are independent, we can say:
\begin{align}
\pr{Y=6} &= \pr{Y=6 | X=6}\\
&= \frac{\pr{(X=6)\cdot(Y=6)}}{\pr{X=6}} \label{eq:5}
\end{align}
Putting \eqref{eq:5} in \eqref{eq:3},
\begin{align}
\pr{A} &= \pr{X=6}\cdot\pr{Y=6}\\
&= \frac{1}{6}\cdot\frac{1}{6}\\
&= \frac{1}{36}
\end{align}
Let event $B$ denote the event where the square of Peehu's dice roll is 36. It is possible if and only if she rolls a 6. Thus,
\begin{align}
\pr{B} = \pr{X=6} = \frac{1}{6} 
\end{align}
As $\pr{B} > \pr{A}$, Peehu has a better chance of getting the number 36 than Apoorv.
\end{document}