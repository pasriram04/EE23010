\let\negmedspace\undefined
\let\negthickspace\undefined
\documentclass[journal,12pt,twocolumn]{IEEEtran}
\usepackage{cite}
\usepackage{amsmath,amssymb,amsfonts,amsthm}
\usepackage{algorithmic}
\usepackage{graphicx}
\usepackage{textcomp}
\usepackage{xcolor}
\usepackage{txfonts}
\usepackage{listings}
\usepackage{enumitem}
\usepackage{mathtools}
\usepackage{gensymb}
\usepackage[breaklinks=true]{hyperref}
\usepackage{tkz-euclide} 
\usepackage{listings}
\usepackage{gvv}
\newtheorem{theorem}{Theorem}[section]
\newtheorem{problem}{Problem}
\newtheorem{proposition}{Proposition}[section]
\newtheorem{lemma}{Lemma}[section]
\newtheorem{corollary}[theorem]{Corollary}
\newtheorem{example}{Example}[section]
\newtheorem{definition}[problem]{Definition}
\newcommand{\BEQA}{\begin{eqnarray}}
\newcommand{\EEQA}{\end{eqnarray}}
\newcommand{\define}{\stackrel{\triangle}{=}}
\theoremstyle{remark}
\newtheorem{rem}{Remark}
\begin{document}

\bibliographystyle{IEEEtran}
\vspace{3cm}

\title{Exemplar - 10.13.2.7}
\author{EE22BTECH11039 - Pandrangi Aditya Sriram$^{*}$% <-this % stops a space
}
\maketitle
\newpage
\bigskip

\renewcommand{\thefigure}{\theenumi}
\renewcommand{\thetable}{\theenumi}

Apoorv throws two dice once and computes the product of the numbers appearing
on the dice. Peehu throws one die and squares the number that appears on it. Who
has the better chance of getting the number 36? Why?\\\solution
Let $A$ be the event that Apoorv's dice have a product of 36, and let $B$ be the event that Peehu's die has a value of 6. Assume that all dice outcomes are equally likely.

Apoorv's dice rolls can be represented as ordered pair of integers. The only case leading to event $A$ is $(6, 6)$. The total number of possible ordered pairs of numbers in the sample space $S$ are $6 \multiply 6 = 36$. Thus, 

\begin{align}
\pr{A} = \frac{n(A)}{n(S)} = \frac{1}{36}
\end{align}

The square of Peehu's dice roll can be 36 if and only if she rolls a 6, which is the only leading to event $B$. Here, the sample space $S$ consists of $6$ values ranging from 1 to 6.

\begin{align}
\pr{B} = \frac{n(B)}{n(S)} = \frac{1}{6} 
\end{align}

As $\pr{B} > \pr{A}$, Peehu has a better chance of getting the number 36 than Apoorv.
\end{document}