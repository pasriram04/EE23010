\let\negmedspace\undefined
\let\negthickspace\undefined
\documentclass[journal,12pt,twocolumn]{IEEEtran}
\usepackage{cite}
\usepackage{amsmath,amssymb,amsfonts,amsthm}
\usepackage{algorithmic}
\usepackage{graphicx}
\usepackage{textcomp}
\usepackage{xcolor}
\usepackage{txfonts}
\usepackage{listings}
\usepackage{enumitem}
\usepackage{mathtools}
\usepackage{gensymb}
\usepackage[breaklinks=true]{hyperref}
\usepackage{tkz-euclide} 
\usepackage{listings}
\usepackage{gvv}
\newtheorem{theorem}{Theorem}[section]
\newtheorem{problem}{Problem}
\newtheorem{proposition}{Proposition}[section]
\newtheorem{lemma}{Lemma}[section]
\newtheorem{corollary}[theorem]{Corollary}
\newtheorem{example}{Example}[section]
\newtheorem{definition}[problem]{Definition}
\newcommand{\BEQA}{\begin{eqnarray}}
\newcommand{\EEQA}{\end{eqnarray}}
\newcommand{\define}{\stackrel{\triangle}{=}}
\theoremstyle{remark}
\newtheorem{rem}{Remark}
\begin{document}

\bibliographystyle{IEEEtran}
\vspace{3cm}

\title{Exemplar - 10.13.2.7}
\author{EE22BTECH11039 - Pandrangi Aditya Sriram$^{*}$% <-this % stops a space
}
\maketitle
\newpage
\bigskip

\renewcommand{\thefigure}{\theenumi}
\renewcommand{\thetable}{\theenumi}

Apoorv throws two dice once and computes the product of the numbers appearing
on the dice. Peehu throws one die and squares the number that appears on it. Who
has the better chance of getting the number 36? Why?\\\solution
Let the random variables $X$ and $Y$ represent Apoorv's dice rolls. Assuming all dice rolls are equally likely,
\begin{align}
p_X(k) &= \frac{1}{6}\\
p_Y(k) &= \frac{1}{6}
\end{align}
where $k \in \{1, 2, 3, 4, 5, 6\}$. There are a total of 36 possible combinations of product $XY$, as shown below:\\
\begin{center}
\begin{tabular}{c c c c c c c}
 \textbf{Product} & \textbf{1} & \textbf{2} & \textbf{3} & \textbf{4} & \textbf{5} & \textbf{6} \\
 \hline
 \textbf{1} & 1 & 2 & 3 & 4 & 5 & 6 \\
 \textbf{2} & 2 & 4 & 6 & 8 & 10 & 12 \\
 \textbf{3} & 3 & 6 & 9 & 12 & 15 & 18 \\
 \textbf{4} & 4 & 8 & 12 & 16 & 20 & 24 \\
 \textbf{5} & 5 & 10 & 15 & 20 & 25 & 30 \\
 \textbf{6} & 6 & 12 & 18 & 24 & 30 & 36 
\end{tabular}
\end{center}
As both dice rolls are independent, and each event in the table is mutually exclusive:
\begin{align}
p_{XY}(m) &= \sum_{kl = m}^{} p_X(k) \cdot p_Y(l)\\
&= \sum_{kl = m}^{} \frac{1}{6} \cdot \frac{1}{6}\\
&= \sum_{kl = m}^{} \frac{1}{36}
\end{align}
Thus, we can calculate the probability distribution by observing the above table:
\begin{align}
p_{XY}(1) = \frac{1}{36}\\
p_{XY}(2) = \frac{2}{36}\\
p_{XY}(3) = \frac{2}{36}\\
p_{XY}(4) = \frac{3}{36}\\
p_{XY}(5) = \frac{2}{36}\\
p_{XY}(6) = \frac{4}{36}\\
p_{XY}(8) = \frac{2}{36}\\
p_{XY}(9) = \frac{1}{36}\\
p_{XY}(10) = \frac{2}{36}\\
p_{XY}(12) = \frac{4}{36}\\
p_{XY}(15) = \frac{2}{36}\\
p_{XY}(16) = \frac{1}{36}\\
p_{XY}(18) = \frac{2}{36}\\
p_{XY}(20) = \frac{2}{36}\\
p_{XY}(24) = \frac{2}{36}\\
p_{XY}(25) = \frac{1}{36}\\
p_{XY}(30) = \frac{2}{36}\\
p_{XY}(36) = \frac{1}{36}
\end{align}
Let the random variable $E$ denote the square of Peehu's dice roll. Thus, the probability distribution function is:
\begin{align}
   p_E(k) = \left \{
   \begin{array}{cc}
      \frac{1}{6} & \quad k \in \{1, 4, 9, 16, 25, 36\} \\
      0 & \quad otherwise
   \end{array}
   \right \}
\end{align}

As $p_{E}(36) > p_{XY}(36)$, Peehu has a better chance of getting the number 36 than Apoorv.

\textbf{Cumulative Distribution Functions:} The cumulative distrubution function $f_X(k)$ is defined as:
\begin{align}
f_X(k) = \sum_{i=-\infty}^{k} p_X(i)
\end{align}
Calculating the Cumulative Distribution Function for Apoorv,
\begin{align}
f_{XY}(1) &= \frac{1}{36}\\
f_{XY}(2) &= \frac{3}{36}\\
f_{XY}(3) &= \frac{5}{36}\\
f_{XY}(4) &= \frac{8}{36}\\
f_{XY}(5) &= \frac{10}{36}\\
f_{XY}(6) = f_{XY}(7) &= \frac{14}{36}\\
f_{XY}(8) &= \frac{16}{36}\\
f_{XY}(9) &= \frac{17}{36}\\
f_{XY}(10) = f_{XY}(11) &= \frac{19}{36}\\
f_{XY}(12) = f_{XY}(13) = f_{XY}(14) &= \frac{23}{36}\\
f_{XY}(15) &= \frac{25}{36}\\
f_{XY}(16) = f_{XY}(17) &= \frac{26}{36}\\
f_{XY}(18) = f_{XY}(19) &= \frac{28}{36}\\
f_{XY}(20) = f_{XY}(21) = f_{XY}(22) = f_{XY}(23) &= \frac{30}{36}\\
f_{XY}(24) &= \frac{32}{36}\\
f_{XY}(k | k \in \{25, 26, 27, 28, 29\}) &= \frac{33}{36}\\
f_{XY}(k | k \in \{30, 31, 32, 33, 34, 35\}) &= \frac{35}{36}\\
f_{XY}(36) &= \frac{36}{36}
\end{align}

The cumulative distribution function for Peehu is, on calculating, it:
\begin{align}
    f_X(k) = \frac{\lfloor\sqrt{k}\rfloor}{6}
\end{align}
for $k \in [0, 36]$
\end{document}